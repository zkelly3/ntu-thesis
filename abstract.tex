\begin{abstractzh}
近年來隨著智慧型汽車相關技術的成熟,車內人臉辨識逐漸受到重視。然而一般的人臉偵測模型尚無法在車內人臉辨識上獲得較好的結果。其原因我們推測為車內人臉辨識會碰到較多在不同光照下的例子,而一般的人臉偵測模型著重於大多數情境下的表現結果,經常忽略在較極端光照下的例子所致。在本論文中我們試著消除不同光照對圖片造成的影響。我們提出一個訓練架構來訓練出能夠對輸入資料在偵測前進行正規化處理的正規器,並將此正規器和人臉偵測器接在一起進行端對端訓練優化。我們的方法在基線做得最差的測試情境中相比基線結果進步了 47.27\%。

\bigbreak
\noindent \textbf{關鍵字:}{\, \makeatletter \@keywordszh \makeatother}
\end{abstractzh}

\begin{abstracten}
As intelligent vehicle technologies become mature, in-car face detection gradually draws everyone's attention. However, general face detection models have yet to perform good when it comes to in-car face detection. We guess the reason is that while in-car face detection has to deal with more cases from various lighting situations, general face detection tends to focus on performing good on major cases and often ignores cases under extreme lighting situations. In this thesis, we tried to remove the effect lighting had on images. We proposed a training architecture to train a normalizer to normalize input images before getting detected, jointed it with a face detector and did end-to-end training for optimization. Our method outperformed baseline by 47.27\% in the test scenario that baseline performed worst on.

\bigbreak
\noindent \textbf{Keywords:}{\, \makeatletter \@keywordsen \makeatother}
\end{abstracten}
