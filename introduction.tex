\chapter{緒論}
\label{c:intro}

人臉偵測一直以來都是深度學習和電腦視覺領域中一個重要的議題。經過了幾年的研究,人臉偵測也漸趨成熟,得以被應用在更多領域中。而其中一個新興的領域便是智慧型汽車。汽車駕駛和我們的生活息息相關,一直以來各大車廠以及研究人員都在不斷研究如何讓駕駛汽車變得更加方便和安全,因而研發出各項技術,如智慧駕駛輔助、自動駕駛等等。隨著這些技術的成熟,車內影像的應用也逐漸受到關注,而對車內人臉的偵測便是一項重要的核心技術。在車內進行人臉偵測能夠協助定位出駕駛與乘客的臉部位置,並能夠接著進行後續的其他應用,如進行身分辨識、偵測駕駛臉部朝向等等,能夠使駕駛汽車更加安全。

然而在現階段,人臉偵測的研究尚無法直接應用於智慧型汽車這個領域。在測試下我們發現,一般的人臉偵測模型做在車內人臉上的表現不如我們預期。車內人臉偵測的結果對後續的相關應用十分重要,因此我們認為如何改善偵測的結果是相當值得研究的。在經過觀察之後我們發現,車內人臉偵測和一般人臉偵測相比有更多在多樣光照下的例子。一般人臉偵測的研究大多著重於提升大多數情境下的表現結果,經常忽略在較極端光照下的例子,因此其在車內人臉的偵測上便表現得較不如預期。

在本研究中,我們在進行人臉偵測前用正規器先對輸入圖片進行正規化處理 (Normalization),並提出了針對此正規器的訓練架構。我們希望能透過消除光照對輸入圖片所造成的差異,來改善模型在人臉偵測上的表現結果。

在接下來的章節中我們會陸續提到以下內容:
在第~\ref{c:related}章,我們會介紹和本研究相關的其他研究,包含了人臉偵測、低光照下的人臉偵測和影像增強。在第~\ref{c:method}章,我們會先闡述我們確定本研究中目標的過程,然後詳細說明在本研究中所使用的方法細節。在第~\ref{c:experiment}章,我們會說明我們在訓練和測試時使用的資料集和實驗設定,展示視覺和數據上的結果,並以實驗說明我們的方法中各個細節對實驗結果造成的影響。而在第~\ref{c:conclusion}章,我們會對本研究做出結論並提出未來努力的目標。
