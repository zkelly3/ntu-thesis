\chapter{緒論}
\label{c:intro}

人臉偵測一直以來都是深度學習和電腦視覺領域中一個重要的議題。經過了幾年的研究,人臉偵測也漸趨成熟,得以被應用在更多領域中。而其中一個新興的領域便是智慧型汽車。近年來人們不斷研究如何讓駕駛汽車變得更加方便和安全,因而研發出各項技術,如智慧駕駛輔助、自動駕駛等等。隨著這些技術的成熟,車內影像的應用也逐漸受到關注,而對車內人臉的偵測便是一項重要的核心技術。

然而在現階段,人臉偵測的研究尚無法直接應用於智慧型汽車這個領域。在測試下我們發現,一般的人臉偵測模型做在車內人臉上的表現不如我們預期。車內人臉的偵測結果對後續的相關應用十分重要,因此我們認為如何改善它的結果是相當值得研究的。經過觀察我們發現,車內人臉偵測和一般人臉偵測相比有更多在不同光線照射下的例子。一般人臉偵測的研究大多著重於提升大多數情境下的表現結果,經常忽略在較極端光照下的例子,因此其在車內人臉的偵測上便表現得較不如預期。

在本研究中,我們在過人臉偵測器前先對輸入資料進行正規化處理 (Normalization),並期待能透過消除不同光線照射對輸入資料所造成的差異,來改善模型在人臉偵測上的表現結果。

在接下來的章節中我們會陸續提到以下內容:
在第~\ref{c:related}章,我們會介紹和本研究相關的其他研究,包含了人臉偵測、低光照下的人臉偵測和影像增強。在第~\ref{c:method}章,我們會詳細介紹在本研究中所使用的方法。在第~\ref{c:experiment}章,我們會說明我們在訓練和測試時使用的資料集和設定,展示視覺和數據上的結果,並以實驗說明我們的方法中各個細節對結果造成的影響。而在第~\ref{c:conclusion}章,我們會對本研究做出結論並提出未來努力的目標。
