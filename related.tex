\chapter{相關研究}
\label{c:related}

\section{人臉偵測}
人臉偵測在電腦視覺領域是個發展成熟的議題,其因能被應用在諸多領域上而有很大的重要性。從以前到現在,人們使用各種不同的方式來解決這個議題。由 P. Viola~和M. J. Jones\cite{viola2004robust} 所提出的 Viola-Jones 目標檢測框架結合了積分圖 (Integral Image)、哈爾特徵 (Haar Feature)、自適應增強 (AdaBoost) 學習、將數個弱分類器級聯 (Cascade) 等概念,率先做到了實時性 (Real-Time) 高精度人臉偵測。而在卷積神經網路興起後,大家對人臉偵測的研究又更加熱烈。由 S. Zhang~\cite{zhang2017faceboxes} 所提出的 FaceBoxes 藉由在卷積層使用較大的步伐 (Stride) 快速將輸入縮小,在盡量不影響結果下減少輸出所需的頻道數,並搭配 Faster R-CNN~\cite{ren2015faster} 中核心的 RPN (Region Proposal Network) 網路和錨點 (Anchor) 的機制來做到高精度實時性的人臉偵測;由 J. Li~\cite{li2019dsfd} 所提出的 DSFD 強化圖片中被擷取的特徵並利用這兩組特徵來算出比單一一組特徵更準確的臉部位置;由 V. Bazarevsky~\cite{bazarevsky2019blazeface} 所提出的 BlazeFace 則使用了輕量化的網路和需要的運算處理較低的架構等,使人臉偵測能夠進一步被用於行動裝置的相關應用上。

\section{低光照下的人臉偵測}
人臉偵測並非只被應用於正常光照下的情境,有時候我們也會需要處理夜間等較有挑戰性的情境,而大家對於這樣的情境也各有不同的解決方案。S. W. Cho~\cite{cho2018face} 使用低光照下的圖片作為訓練資料,試圖使訓練時的情境接近測試時的情境以提升人臉偵測的準確率;M. K. Bhowmik~\cite{bhowmik2011thermal} 和 J. Kang~\cite{kang2015face} 分別使用了熱紅外光攝影機和近紅外光攝影機獲取非可見光照射下的資料,以避開低光照對圖片造成的影響;也有一些研究~\cite{cho2018face, li2017real, yang2020advancing}選擇對圖片進行增強,試圖還原在低光照下損失的色彩與細節以提升準確率。

\section{影像增強}
影像增強也是一個非常熱門的議題,人們透過增強圖片來還原圖片在低光照下損失的色彩與細節,讓圖片能被進行其他後續處理。較為傳統並廣為人知的作法有直方圖均衡化 (Histogram Equalization)~\cite{gonzales2002digital} 和伽瑪校正 (Gamma Correction)~\cite{gonzales2002digital}等,這些方法主要透過拉高像素間的對比來達到強調細節的效果。過去研究人員模仿人體視覺系統發展出視網膜增強算法~\cite{land1977retinex},近期也有研究受到這個算法的啟發。由 L. Shen~\cite{shen2017msr} 提出的 MSR-net 認為視網膜增強算法的架構和卷積神經網路相似,並將其演算法轉換成卷積神經網路;C. Wei~\cite{wei2018deep} 的研究將圖片分為光照 (Illumination) 和反射 (Reflectance) 兩部分,把光照部分增強後再和反射部分相乘得到最終結果。除此之外,還有其他基於深度學習對影像增強的研究。J. Cai~\cite{cai2018learning} 的研究將圖片分為低頻和高頻部分分別進行增強,融合兩個結果後又進行第二次增強來獲得最終結果;由K. G. Lore~\cite{lore2017llnet} 所提出的 LL-net 則使用自編碼器 (Autoencoder) 對低光照圖片進行去噪和增強。
