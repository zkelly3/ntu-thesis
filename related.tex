\chapter{相關研究}
\label{c:related}

\section{人臉偵測}
\cite{Rowe:2005:ASR}人臉偵測在電腦視覺是個發展成熟的議題,其能被應用在諸多領域上,因此有很大的重要性。從以前到現在,人們使用各種不同的方式來解決這個議題。Viola-Jones 人臉偵測器結合了積分圖 (Integral Image)、哈爾特徵 (Haar Feature)、自適應增強 (AdaBoost) 學習、將數個弱分類器級聯 (Cascade) 等概念,率先做到了實時性 (Real-Time) 高精度人臉偵測。在卷積神經網路興起後,大家對人臉偵測的研究又更加熱烈。FaceBoxes 藉由在卷積層使用較大的步伐 (Stride) 快速將輸入縮小,在盡量不影響結果的情況下減少輸出所需的頻道數,並搭配 Faster R-CNN 中核心的 RPN (Region Proposal Network) 網路和錨點 (Anchor) 的機制來做到高精度實時性的人臉偵測;DSFD 強化圖片中被擷取的特徵並利用這兩組特徵來算出比單一一組特徵更準確的臉部位置;BlazeFace 則使用了輕量化的網路和需要的運算處理較低的架構等,使人臉偵測能夠進一步被用於行動裝置的相關應用上。

\section{低光照下的人臉偵測}
人臉偵測並非只被應用於正常光照下的情境,有時候我們也會需要處理夜間等較有挑戰性的情境,而大家對於這樣的情境也各有不同的解決方案。有些人 (Se Woon Cho) 使用低光照下的圖片作為訓練資料,試圖使訓練時的情境接近測試時的情境以提升偵測的準確率;也有人(Mrinal Kanti Bhowmik 和 Jinwoo Kang) 使用了紅外光攝影機獲取非可見光照射下的資料,避開了低光照對圖片造成的影響;其他人則選擇了對圖片進行增強,試圖還原在低光照下損失的色彩與細節。

\section{影像增強}
影像增強也是一項非常熱門的應用,其大量被用於加強低光照下的圖片使損失的色彩與細節得以被還原。較為傳統並廣為人知的作法包含了直方圖均衡化、珈瑪校正和視網膜增強算法等,前兩者主要透過拉高像素間的對比來達到強調細節的效果,而第三者透過模擬人眼視網膜以計算出圖片增強後的結果。其餘方法多運用深度學習訓練出增強圖片的模型。MSR-net 受啟發於帶色彩恢復的多尺度視網膜增強算法並將其演算法轉換成卷積神經網路;Jianrui Cai 將圖片分為低頻和高頻並分別進行增強,融合兩個結果後再進行第二次增強得到最後結果;Chen Wei 把圖片分為光照和反射兩部分,把光照部分增強後再和反射部分相乘得到最終結果。
